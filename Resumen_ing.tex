% RESUMEN INGLES
% PAGINA 4 DE TESIS
% ENUMERADA ROMANO

\newpage % Que inicie en pagina nueva

\addcontentsline{toc}{chapter}{Abstract} % Para agregar a Tabla de Contenido

\textbf{Abstract} of the thesis presented by \textbf{Miguel Ángel Murillo Escobar}, as a partial requirement to obtain the DOCTOR IN SCIENCE degree in ELECTRIC with orientation in CONTROL, of the program of Master and Doctorate in Science and Engineering of the Autonomous University of Baja California. Ensenada, Baja California, Mexico. May 2015.  

\vspace{0.3cm}

\begin{center}
  \textbf{DESIGN OF A CHAOTIC CIPHER ALGORITHM \\AND ITS IMPLEMENTATION IN MICROCONTROLLER \\FOR EMBEDDED APPLICATIONS}
\end{center}  

\vspace{0.3cm}

Abstract approved by:

\vspace{0.1cm}

\begin{flushright}
	\begin{tabular}{c}
	\includegraphics[scale=0.3]{VOTOS/FIRMA_CESAR.pdf} \\
	\hline
	\textbf{Dr. César Cruz Hernández} \\
	\textit{Thesis director}
	\end{tabular}
\end{flushright}

\vspace{0.3cm}

In this PhD thesis, a cryptographic algorithm based on chaos is designed and implemented in a 32-bit microcontroller for secure data transmission and storage in embedded applications. \\

Firstly, the implementation of two chaotic systems are analyzed: Lorenz system and logistic map. We chose logistic map due it is a discrete system by nature, it has better performance, and it requires less implementation resources than Lorenz system; in addition, chaos existence is verified with maximum Lyapunov exponent. Then, the cipher algorithm is designed by using just one round of the confusion and diffusion architecture.  Finally, the proposed encryption algorithm is applied in three different cases to prove its versatility: \textit{1:~Color image} (at software level), \textit{2:~Alphanumeric text} y \textit{3:~Fingerprint template}. \\ 

Several security analysis such as decryption error, key space, secret key sensitivity, plain input sensitivity, histograms, correlation, information entropy, floating frequency, N-grams, autocorrelation, random statistic, test with FIPS-140-2, encryption time, and hardware analysis as memory used, communication ports, frequency system, and cipher time, verify and validate the proposed cipher algorithm to be used in real-time embedded systems. \\

Some applications of interest are military, medicine, telecommunications, biometric, computation, industry, electronic payment, personal information, into others. \\

\vspace{0.4cm}
\textbf{Keywords:} chaotic cipher, security analysis, microcontroller, embeeded applications.  
