% Apendice A

\chapter{Programa MatLab: cifrado de imagen a color RGB}

En esta apéndice, se presenta el software de cifrado propuesto en este trabajo doctoral aplicado a imágenes a color RGB. Se realizo con lenguaje C en MatLab. Los software de cifrado de texto y de plantilla dactilar que se implementaron en microcontrolador son similares al que se presenta en este apéndice. 

\begin{lstlisting}
clear all;  % Limpia variables.
format long; % Variables con 15 decimales de precision.
tic % Mide tiempo de proceso. 

%% Clave secreta:
CLAVE = '1234567890ABCDEF1234567890ABCDEF';
TAMANO_CLAVE = size(CLAVE,2);

%% Leer las matrices RGB de imagen clara:
P = imread('lena_png_512.png'); 
[FIL COL MAT]=size(P)

%% Manejo de clave secreta Tabla 6.1
% Separar en 4 secciones:
SA1 = CLAVE(1:8);
SA2 = CLAVE(9:16);
SX1 = CLAVE(17:24);
SX2 = CLAVE(25:32);

% Convertir de hexadecimal a decimal 
dec_SA1 = hex2dec(SA1);
dec_SA2 = hex2dec(SA2);
dec_SX1 = hex2dec(SX1);
dec_SX2 = hex2dec(SX2);

% Determinar el valor a sumar en las C.I. y P.C.:
A = dec_SA1 / (4294967296 + 1);
B = dec_SA2 / (4294967296 + 1);
C = dec_SX1 / (4294967296 + 1);
D = dec_SX2 / (4294967296 + 1);

%% Iteracion del mapa logistico 2:
PC_L2 = mod(A + B,1) * 0.001;
x2 = mod(C + D,1);

a2 = 3.999 + PC_L2; % Parametro de control de logistico 2
LOG_2(1) = x2;      % Condicion inicial de logistico 2

R=1000;    % Iteraciones de mapa logistico 2
for n=1:R
    LOG_2(n+1)=a2*LOG_2(n)*(1-LOG_2(n));
end

LOG_2=mod(LOG_2*1000,1); % Optimizacion de secuencias caoticas.

%% Suma de pixeles de imagen clara con 50 datos caoticos de logistico 2:
a_sumar_32k=im2double(P); %Transformar de 0-255 a 0-1

if(MAT == 1)    % Para imagen a escala de grises
    sum_vec_32k=0;
    for k=1:FIL
    sum_vec_32k =sum_vec_32k + a_sumar_32k(k,:,1);
    end    
else  
    sum_vec_32k=0;
    for k=1:FIL
    sum_vec_32k =sum_vec_32k + a_sumar_32k(k,:,1) + a_sumar_32k(k,:,2) + 
    a_sumar_32k(k,:,3);
    end
end

S=0;
for j=1:size(sum_vec_32k,2)
   S=S+sum_vec_32k(j); 
end

% Agregar los ultimos 50 datos de mapa logistico 2:
F=0;
for n=1:50
    F = F + LOG_2(1001-n);
end

v1 = mod((S*1000) + F,1);   % Valor entre 0-1.
v2 = 1 + round(v1 * 253);   % Valor entre 1-254.
Z = v2 / 255;               % Valor de Z.
                             
%% Iteracion del mapa log\'istico 1:
PC_L1 = mod(A + B + Z,1) * 0.001;
x1 = mod(C + D + Z,1);

a1 = 3.999 + PC_L1; % Parametro de control de logistico 1
LOG_1(1) = x1;      % Condicion inicial de logistico 1

T=5000;     % Iteraciones de mapa logistico 2
for n=1:T
    LOG_1(n+1)=a1*LOG_1(n)*(1-LOG_1(n)); % mapa logistico
end

%% Determinar local para almacenar Z en criptograma:
if(MAT == 1) % Para imagen a escala de grises.
    I = round((LOG_2(R-10) * (FIL-1))+1);
    J = round((LOG_2(R-100)* (COL-1))+1);
    K = 1;    
else         % Para imagen a color.
    I = round((LOG_2(R-10) * (FIL-1))+1);
    J = round((LOG_2(R-100)* (COL-1))+1);
    K = round((LOG_2(R-200) * (MAT-1))+1);
end

%% Sub-secuencia para confusion de filas:
% Seleccion de los ultimos datos caoticos.
RE(1:FIL)=round((LOG_1((T+1)-FIL:T).*(FIL-1))+1);

% Limpia de variables
valor_r_32k = 0;
vect_r_32k = 0;
num_no_est_32k = 0;

% Busca posiciones de repetidos y los pone en un vector
bus_32k(1)=RE(1);
s=1;
for b=2:FIL
    en=1;
    bus_32k(b)=RE(b);
    for j=1:b-1
       if(bus_32k(b) == bus_32k(j) && en == 1) 
            valor_r_32k(s)=bus_32k(j);
            vect_r_32k(s)=b;
            s=s+1;
            en=0;
       end
    end
end

% Buscar numeros que no esten en RE 
tam_v_32k=size(valor_r_32k,2);
preg_num=1;
saltar=0;
no=1;
for v=1:FIL
    for vv=1:tam_v_32k
          if(saltar == 0 && preg_num == valor_r_32k(vv))
              saltar=1;
              %entra aqui si esta el numero (no se puede usar)
           end
    end
    % Pregunta si esta en RE
    for vvv=1:FIL
        if(preg_num == RE(vvv))
            saltar=1; % si esta entra.
        end      
    end
    % Si no esta en repetidos y en RE, entra
    if(saltar == 0)
        num_no_est_32k(no)=preg_num; % Guarda n\'umero.
        no=no+1;
    end
    saltar=0;   
    preg_num=preg_num+1;
end

tam_vect_32k=size(vect_r_32k,2);

% Se actualiza RE (tiene todos los espacios)
en_dos=round(tam_vect_32k/2);
cambio=1;
S=1;
for s=1:tam_vect_32k
    if(cambio == 1)
     RE(vect_r_32k(s))=num_no_est_32k(S);
     cambio=0;
    else
     RE(vect_r_32k(s))=num_no_est_32k(S+en_dos);
     S=S+1;
     cambio=1;        
    end
end

%% Sub-secuencia para confusion de columnas:
CO = round((LOG_1((T+1)-COL-FIL:T-FIL).*(COL-1))+1);

% Lipmia de variables
valor_re_32k = 0;
vect_re_32k = 0;
num_no_esta_32k = 0;

% Busca posiciones de repetidos y los pone en un vector
busq_32k(1)=CO(1);
s=1;
for b=2:COL
    en=1;
    busq_32k(b)=CO(b);
    for j=1:b-1
       if(busq_32k(b) == busq_32k(j) && en == 1) 
            valor_re_32k(s)=busq_32k(j);
            vect_re_32k(s)=b;
            s=s+1;
            en=0;
       end
    end
end

% Buscar numeros que no esten en CO 
tam_ve_32k=size(valor_re_32k,2);
preg_nume=1;
salta=0;
no=1;
for v=1:COL
    for vv=1:tam_ve_32k
          if(salta == 0 && preg_nume == valor_re_32k(vv))
              salta=1;
              %entra aqui si esta el numero (no se puede usar)
           end
    end
    % Pregunta si esta en CO
    for vvv=1:COL
        if(preg_nume == CO(vvv))
            salta=1; % si esta entra.
        end      
    end
    % Si no esta en repetidos y en CO, entra
    if(salta == 0)
        num_no_esta_32k(no)=preg_nume; % Guarda numero.
        no=no+1;
    end
    salta=0;   
    preg_nume=preg_nume+1;
end

tam_vecto_32k=size(vect_re_32k,2);

% Se actualiza CO (tiene todos los espacios)
en_dos=round(tam_vecto_32k/2);
cambio=1;
S=1;
for s=1:tam_vecto_32k
    if(cambio == 1)
     CO(vect_re_32k(s))=num_no_esta_32k(S);
     cambio=0;
    else
     CO(vect_re_32k(s))=num_no_esta_32k(S+en_dos);
     S=S+1;
     cambio=1;        
    end
end

%% Sub-secuencia para difusion:
M = mod((LOG_1((T+1)-5000:T).*(1000)) + Z,1); 

% Construir una matriz de M x N
inc=1;
for k=1:FIL
    for kk=1:COL
        MATRIZ_M(k,kk)=M(inc);
        inc=inc+1;
        if (inc == 5001)
            inc=1;
        end
    end
end

%% Proceso de cifrado: una ronda de confusion-difusion
P_DOUBLE=im2double(P); % Se transforma P de 0-255 a 0-1

if(MAT == 1) % Para imagenes a escala de grises.
    E_DOUBLE(:,:,1)=P_DOUBLE(RE(:),CO(:),1) + MATRIZ_M(:,:);
else        % Para imagen a color.
    E_DOUBLE(:,:,1)=P_DOUBLE(RE(:),CO(:),1) + MATRIZ_M(:,:); 
    E_DOUBLE(:,:,2)=P_DOUBLE(RE(:),CO(:),2) + MATRIZ_M(:,:); 
    E_DOUBLE(:,:,3)=P_DOUBLE(RE(:),CO(:),3) + MATRIZ_M(:,:);
end

%% Agregar valor de Z en imagen cifrada
E_DOUBLE(I,J,K)= Z; % Agrega Z 
E_DOUBLE = mod(E_DOUBLE,1); % Valores entre 0 - 1
E = im2uint8(E_DOUBLE); % Transformar a 0 - 255

toc % Termina de medir tiempo de calculo.

%% Guardar imagen cifrada y desplegar histogramas.
imwrite(E, 'E.png'); %Guarda imagen cifrada
% Muestra imagen clara y cifrada
figure;subplot(1,2,1);imshow(P)
title('Imagen clara')
subplot(1,2,2);imshow(E)
title('Imagen cifrada')

% Histogramas
figure;
if(MAT == 1) % Para imagenes a escala de grises.
    matriz_roj=E(:,:,1);
    subplot(2,1,1);imshow(matriz_roj)
    title('Imagen a ROJO')
    subplot(2,1,2);imhist(matriz_roj)
    title('Histograma Imagen ROJO')
else    % Para imagen a color.
    matriz_roj=E(:,:,1);
    matriz_verd=E(:,:,2);
    matriz_azu=E(:,:,3);    
    
    subplot(3,2,1);imshow(matriz_roj)
    title('Imagen a ROJO')
    subplot(3,2,3);imshow(matriz_verd)
    title('Imagen a VERDE')
    subplot(3,2,5);imshow(matriz_azu)
    title('Imagen a AZUL')
    % Graficar Histograma
    subplot(3,2,2);imhist(matriz_roj)
    title('Histograma Imagen ROJO')
    subplot(3,2,4);imhist(matriz_verd)
    title('Histograma Imagen VERDE')
    subplot(3,2,6);imhist(matriz_azu)
    title('Histograma Imagen AZUL')   
end
\end{lstlisting} 