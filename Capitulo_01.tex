% CAPITULO I

\chapter{Introducción}
En la actualidad, miles de kilobytes de datos privados se transmiten diariamente a través de medios de comunicaciones inseguros (como internet, redes de computadoras, sistemas de comunicaciones, etc.) en distintas áreas que son susceptibles a robo de información, principalmente para fines de estafas, fraudes o bélicos (como en comercio electrónico, informática, sistemas biométricos, telemedicina, sismología, información personal, sistemas embebidos,  entre otros), por lo que surge la necesidad de proteger la información en su almacenamiento y transmisión. \\

En los últimos años, los sistemas \textit{no lineales} y especialmente los \textit{sistemas caóticos} han generado mucho interés en muchas áreas científicas, por ejemplo en física, matemáticas, economía, biología, meteorología, ciencias sociales, medicina y astronomía, ya que muchos de sus sistemas se pueden describir de forma caótica. Otra aplicación muy interesante del caos, es en las comunicaciones seguras y la \textit{criptografía}. El objetivo principal de la criptografía es garantizar la confidencialidad de la información almacenada o transmitida, es decir, garantizar que sólo el personal autorizado tenga acceso a dicha información.\\   

Por otra parte, el criptoanálisis es la ciencia que se ocupa de corromper un sistema criptográfico y determinar el mensaje original a partir del mensaje cifrado o la clave de cifrado. Para ello, se utilizan análisis matemáticos y estadísticos que se conocen como ataques criptoanalíticos, los cuales varían según el método criptográfico implementado. Por tanto, un sistema criptográfico debe resistir los distintos ataques criptoanalíticos conocidos en la actualidad para que se considere seguro criptográficamente (Sec. 3.3). Además de la seguridad criptográfica, el sistema criptográfico debe ser eficiente para cifrar información a alta velocidad y que cada cifrado genere características similares de seguridad. \\   

Desde 1930, máquinas mecánicas y electromecánicas de cifrado (basadas en criptografía clásica) se hicieron presente en la Segunda Guerra Mundial con el objetivo de establecer comunicaciones seguras a nivel militar, como la máquina electromecánica de rotores llamada Enigma utilizada por alemanes. Mientras que a nivel civil, comienzan en las décadas de los 70, donde se desarrolla, estandariza e implementa el famoso algoritmo criptográfico \textit{DES} (del inglés, \textit{Data Encryption Standard}) para aplicaciones de comunicación electrónica segura para empresas financieras. Actualmente este algoritmo de cifrado es considerado inseguro para muchas aplicaciones debido a que utiliza una clave de cifrado de 56 bits, debido a que no resiste al criptoanálisis de un ataque de búsqueda exhaustiva donde se prueba cada posible clave para determinar el mensaje claro \cite{E_1998}.\\

La metodología de los algoritmos criptográficos actuales (criptografía convencional) se basan en propiedades algebraicas y numéricas, como estructura de Feistel del \textit{TDES} (del inglés, \textit{Triple Data Encryption Standard}) y arquitectura de permutación y difusión del \textit{AES} (del inglés, \textit{Advanced Encryption Standard}). Frente a estos paradigmas, el \textit{caos} pertenece al campo de los sistemas dinámicos \textit{no lineales}, que manifiesta comportamientos ``pseudoaleatorios'' determinísticos, por lo que se clasifica en la criptografía caótica. Actualmente, se reconoce que los sistemas caóticos (representados matemáticamente por ecuaciones diferenciales o en diferencias no lineales) poseen muchas propiedades criptográficas como alta sensibilidad a condiciones iniciales y a parámetros de control, mezcla de datos, entro otros (Sec.~4.5), lo cual, hace que el caos sea muy interesante y efectivo para el cifrado de información. \\

A partir del trabajo de Pecora y Carroll en 1990, donde se reportó la sincronización de dos sistemas caóticos \cite{PyC_1990,PyC_1991}, en los últimos años, se reportaron en la literatura varias técnicas de cifrado caótico \textit{analógico} con base en la sincronización de caos, como \textit{enmascaramiento caótico}, \textit{conmutación caótica}, \textit{modulación caótica}, entre otros, ver por ejemplo \cite{CEtAl_1992, CyO_1992, CyO_1993, CEtAl_1993, MyZ_1996, GEtAl_2000, C_2004, PEtAl_2004, LyC_2004, LyC_2005, CEtAl_2005, LEtAl_2005, CyR_2006, PEtAl_2007, AEtAl_2008a, LEtAl_2009a, LEtAl_2009b, GEtAl_2009, SEtAl_2010a, CEtAl_2012, AEtAl_2012b}, donde la idea fundamental consiste en utilizar un sistema caótico en régimen caótico para generar una secuencia de banda ancha pseudoaletoria y la combina con el mensaje claro para producir una señal de aspecto incomprensible que se transmite a través de un canal inseguro. Mediante la sincronización, el sistema receptor produce la misma señal pseudoaleatoria para restarla con la señal que recibe del canal inseguro y recuperar el mensaje claro (ver figura~1.1). \\

\begin{figure}[!htbp] % Por todos lo medios (!) aqui (h), superior (t), o inferior (b) o flotante (p)
	\center
	\includegraphics[scale=0.45]{FIG_CAP_01/SCH_CIFRADO_ANAL.pdf}  
	\caption{Esquema de cifrado \textit{analógico} unidireccional empleando \textit{enmascaramiento caótico} con dos sistemas de Lorenz \cite{CEtAl_1993}.}
\end{figure}

Estos sistemas criptográficos son implementados mediante circuitos analógicos, tanto en el transmisor como en el receptor, empleando resistencias, condensadores, bobinas y amplificadores operacionales básicamente (ver figura~1.2). Los valores de las resistencias y condensadores determinan los valores de los parámetros de control de los sistemas caóticos y representan la \textit{clave secreta} de cifrado. En el mercado, estos componentes se consiguen con una exactitud de 0.1\% como máximo, lo que ocasiona un pequeño desajuste entre los valores de los parámetros del transmisor y el receptor (además, los componentes presentan desgaste y el valor nominal varía con el tiempo). Debido a la alta sensibilidad que presentan los sistemas caóticos a los parámetros de control, este factor influye de forma negativa en dos aspectos de suma importancia \cite{OEtAl_2012}:
\begin{enumerate}
\item \textit{Recuperación del mensaje claro en el receptor}. Por el desajuste natural de los componentes electrónicos, la sicronización caótica entre transmisor y receptor presenta un error que incrementa con el tiempo, es decir, $x \approxeq x^{R}$ en figura~1.1. Por este motivo, la recuperación del mensaje claro en el receptor es inexacta, es decir, $\widehat{m} \approx m$. Este problema influye de forma negativa en aplicaciones donde se requiere que el mensaje descifrado sea exactamente el mismo que el mensaje claro, por ejemplo en la milicia, medicina, telecomunicaciones, biometría, sismología, entre otros. 
\item  \textit{Seguridad criptográfica}. Un gran número de resultados criptoanalíticos a sistemas criptográficos basados en caos \textit{analógico} se reportaron en la literatura y se mostró que no son suficientemente seguros desde el punto de vista \textit{criptográfico}, ya que presentan problemas como baja sensibilidad a la clave secreta (tal vez el problema más serio \cite{WEtAl_2004a}), espacio de claves reducido, fácil estimación de parámetros, fácil estimación de la señal portadora del sistema caótico maestro en técnicas de enmascaramiento caótico, extracción del texto claro de forma directa mediante filtrado, análisis de potencia, análisis de periodo corto, entre otros \cite{ZEtAl_2006, LEtAl_2007, QEtAl_2009, PyL_2011}. Aunque en la literatura se presentaron ciertas contramedidas, los sistemas criptográficos basados en caos \textit{analógico} son considerados poco seguros en términos criptográficos \cite{LEtAl_2007}, por lo que se requiere de mayor atención en la seguridad del cifrado en este tipo de sistemas de comunicación. \\
\end{enumerate}

Algunos esquemas de cifrado analógico presentados son razonablemente veloces para cifrar información y simples de implementar, sin embargo, la mayor parte de los reportes no incluyen un análisis de seguridad para verificar el nivel de confidencialidad de la información cifrada. En general, se consideró al sistema criptográfico ``seguro'' por el hecho de utilizar un sistema caótico para cifrar información y de acuerdo con Kocarev \cite{K_2001}, el uso de caos \textit{analógico} en un sistema criptográfico no es garantía de seguridad. Para evaluar la seguridad de un sistema criptográfico, todos los ataques criptoanalíticos conocidos deben ser probados sobre el esquema de cifrado \cite{LEtAl_2007}. Cabe mencionar que la seguridad no es el único requerimiento para un esquema de cifrado práctico, también se deben considerar otras propiedades como velocidad de cifrado y robustez al ruido externo.  \\

\begin{figure}[!htbp] % Por todos lo medios (!) aqui (h), superior (t), o inferior (b) o flotante (p)
	\center
	\includegraphics[scale=0.1]{FIG_CAP_01/FOTO_CIFRA_AUDIO.pdf}  
	\caption{Sistema criptográfico basado en caos \textit{analógico} implementado en circuitería electrónica para el cifrado de audio analógico.}
\end{figure}

Por otra parte, la aplicación de cifrado caótico \textit{digital} se ha convertido en un campo emergente con altas expectativas en seguridad, flexibilidad, aplicabilidad y eficiencia criptográfica. Cifrado caótico \textit{digital} no requiere el proceso de sincronización entre transmisor y receptor, por tanto, se evitan problema de seguridad que se presentan en sistemas criptográficos basados en caos \textit{analógico}. Un sistema caótico es implementado en forma \textit{digital} mediante soluciones numéricas si el sistema caótico esta representado por ecuaciones diferenciales (tiempo continuo) no lineales como Lorenz, Chua, Chen, Lu, CNN, Genesio-Tesi y R\"{o}ssler, o el sistema caotico es implementado directamente si esta representado por ecuaciones en diferencias (tiempo discreto) no lineales como logístico 1D, logístico 2D, Hénon y Arnold-cat en una computadora, microcontrolador, FPGA, ASIC u otro sistema digital. Los valores de condiciones iniciales y parámetros de control de los sistemas caóticos (varia la cantidad según el sistema caótico que se utilice y si este es de una, dos o más dimensiones) constituyen la clave secreta (magnitudes digitales), de tal forma que \textit{la dinámica caótica generada en el transmisor y en el receptor son idénticas}. Por tanto, los sistemas criptográficos basados en caos \textit{digital} pueden ser más seguros y prometedores que los sistemas criptográficos basados en caos \textit{analógico} por lo siguiente:

\begin{enumerate}
\item \textit{Recuperación del mensaje claro en el receptor}. En un sistema de cifrado caótico digital se garantiza que $x = x^{R}$ (ver figura~1.1). Por tanto, el mensaje descifrado corresponde de forma exacta al mensaje claro, es decir, $\widehat{m} = m$ en la figura~1.1. 
\item  \textit{Seguridad criptográfica}. La precisión digital de magnitudes digitales puede ser de $10^{-15}$. Considerando esto y el diseño de un algoritmo criptogáfico que genere información cifrada con excelentes propiedades estadísticas de pseudoaleatoriedad, la criptografia basada en caos digital puede presentar mayor seguridad ante diversos ataques criptoanalíticos conocidos en la actualidad \cite{YyH_2010, AEtAl_2012, FEtAl_2013, B_1998, WyT_2012, AyJ_2007, HEtAl_2014, ZEtAl_2014a, MyM_2012, PEtAl_2006, PEtAl_2009, HyX_2010, C_2010, CyC_2011, WEtAl_2012, L_2012, IyC_2013, NEtAl_2014, KEtAl_2007, XEtAl_2011}.  
\item \textit{Flexibilidad}. El algoritmo criptográfico se puede implementar en cualquier sistema digital, por ejemplo en microcontrolador, FPGA, ASIC, computadora, entre otros.
\item \textit{Aplicabilidad}. El cifrado de datos aplica para cualquier tipo de información digital y también, para información analógica (con el uso de un convertidor analógico a digital.)
\item \textit{Eficiencia criptográfica}. El cifrado de información digital es altamente eficiente al considerar costo \textit{vs} velocidad de cifrado, ya que es posible cifrar datos a alta velocidad a un bajo costo.
\end{enumerate}

En los últimos años, en la literatura se reportaron distintas técnicas de cifrado \textit{digital}, por ejemplo, cifrado de flujo basado en PRNG caóticos (generación de números pseudoaleatorios) \cite{YyH_2010, AEtAl_2012, FEtAl_2013}, cifrado caótico basado en la ergodicidad (seccionar en \textit{n} partes la secuencia caótica) \cite{B_1998, WyT_2012}, cifrado en bloque con cajas-S dinámicas caóticas (referente a la S-box del algoritmo AES) \cite{AyJ_2007, HEtAl_2014, ZEtAl_2014a}, entre otros. Algunos algoritmos de cifrado caótico con base a programación en MatLab se reportan en la literatura, para texto en \cite{MyM_2012}, imágenes en \cite{PEtAl_2006, PEtAl_2009, HyX_2010, C_2010, CyC_2011, WEtAl_2012, L_2012, IyC_2013, NEtAl_2014} y datos biométricos en \cite{KEtAl_2007, XEtAl_2011}. Sin embargo, en su mayoría presentan serios problemas de seguridad y han sido vulnerados, ver por ejemplo \cite{WEtAl_2005, WEtAl_2007, ByN_2008, CyS_2009, AyL_2009, LEtAl_2009, REtAl_2010, SEtAl_2010, LEtAl_2011, ZEtAl_2011, LEtAl_2012}.  \\

El uso de caos \textit{digital} tampoco es garantía de seguridad. La seguridad de un algoritmo de cifrado basado en caos \textit{digital} consiste de un espacio de claves suficientemente grande para resistir un ataque exhaustivo, en el diseño de un algoritmo de cifrado que genere texto cifrado con excelentes propiedades estadísticas, que resista ante los ataques criptoanalíticos conocidos y que el algoritmo sea factible para aplicaciones en tiempo real \cite{AEtAl_2009}.  \\

\section{Motivación}
El crecimiento tecnológico de las últimas décadas ha traído múltiples beneficios a la humanidad, principalmente el desarrollo de nuevos sistemas de comunicaciones entre individuos y desarrollo de nuevos sistemas digitales embebidos para la recolección, control y transmisión de datos. Sin embargo, cualquier información transmitida a través de un canal inseguro es comprometida y puede ser robada por intrusos, lo que genera un problema de seguridad.\\ 

La \textit{teoría de caos} se estableció desde 1970 en varias disciplinas científicas, como física, matemáticas, economía, biología, astronomía, ingeniería y química. Los sistemas caóticos tienen muchas propiedades interesantes como ergodicidad, alta sensibilidad a condiciones iniciales, alta sensibilidad a parámetros de control, mezcla de datos, no linealidad, atractores extraños, dinámicas pseudoaleatorias, etc., que en su mayoría, están relacionadas con los requerimientos de Shannon para los procesos de permutación y difusión, para construir un sistema criptográfico seguro \cite{S_1948, S_1949}. Debido a la estrecha relación entre caos y criptografía \cite{ByC_1996, F_1998, AEtAl_1999}, existe un gran interés en construir esquemas de comunicación segura con el uso de caos, para proteger información confidencial y evitar robo o acceso ilegal a la información en su transmisión y en su almacenamiento, en áreas como la milicia, telemedicina, biometría, informática, industria, información personal, entre otros. \\

De acuerdo con el principio de Kerckoff \cite{P_2011}, la seguridad del sistema criptográfico debe recaer sobre la clave secreta y no sobre el proceso de cifrado, ya que este se considera de dominio público (criptografía moderna). Esto no significa desatender el proceso de cifrado, al contrario, este debe ser un proceso complejo. Por tanto, la clave de cifrado es de suma importancia, la cual, debe estar claramente definida y debe tener las combinaciones suficientes para resistir un ataque exhaustivo considerando el poder computacional actual.  \\

De acuerdo con lo mencionado al inicio de este capítulo, los sistemas criptográficos basados en caos \textit{analógico} tienen serios problemas de seguridad en el sentido criptográfico, ya que presentan desventajas como \textit{problemas para la recuperación del mensaje claro de forma exacta}, un \textit{espacio de claves reducido}, \textit{baja sensibilidad a la clave secreta}, \textit{fácil estimación del parámetro de control}, \textit{fácil estimación de la señal portadora}, entre otros. Estos inconvenientes ocasionan que los sistemas criptográficos basado en caos \textit{analógico} no sean criptográficamente confiables para su uso en aplicaciones prácticas.  \\

Los problemas de seguridad que presenta la criptografía basada en caos \textit{analógico} se ven grandemente reducidos con el empleo de la tecnología digital, por lo que, la criptografía basada en caos \textit{digital} es un área de investigación emergente de los últimos años, con altas expectativas en eficiencia y seguridad criptográfica. Es de mucho interés público y científico, el diseñar e implementar nuevos sistemas criptográficos no convencionales, principalmente con el uso de sistemas caóticos, que sean seguros contra los ataques criptoanalíticos más poderosos reportados en la literatura y que en los últimos años han quebrantado un sin número de algoritmos de cifrado caótico digital, y que además sean eficientes para aplicaciones embebidas, principalmente para implementaciones biométricas con el uso de microcontroladores, donde los datos biométricos personales son susceptibles a robo de identidad en esquemas presentados recientemente en la literatura (ver sección~9.1).  \\

Muchos de los primeros sistemas criptográficos basado en caos analógico presentados en la literatura no eran conscientes de los conceptos y estándares \textit{criptográficos}, por lo que en su mayoría resultaron criptográficamente inseguros e ineficientes. Actualmente, muchos sistemas criptográficos basado en caos digital que fueron propuestos en la literatura, tuvieron en cuenta muchos principios criptográficos, sin embargo, no han sido suficientes y siguen siendo inseguros e ineficientes, principalmente por problemas en el diseño del algoritmo criptográfico y el mal uso del caos. Por tanto, existe una gran necesidad de continuar en la investigación, desarrollo y construcción de nuevos sistemas criptográficos caóticos que sean altamente seguros y eficientes para aplicaciones en tiempo real.
             
\section{Objetivos y alcances de la tesis}
Debido al gran interés que ha surgido en la comunidad científica sobre la aplicación de algoritmos criptográficos no convencionales, en particular el cifrado caóticos, en la realización de este trabajo de tesis doctoral se planteó alcanzar el siguiente \textit{objetivo general}: \\ 

{\large \textbf{Diseñar e implementar un algoritmo de cifrado caótico digital altamente seguro y eficiente, para su aplicación en sistemas embebidos.}} \\

Que para cumplir con el objetivo general, se plantea alcanzar los siguientes\textit{ objetivos particulares}:

\begin{enumerate}
\item Determinar el sistema caótico a utilizar para ser implementado en microcontrolador, con base a su desempeño y recursos físicos: sistema de Lorenz o mapa logístico.
\item Diseñar un algoritmo criptográfico con base en: clave simétrica, arquitectura de confusión y difusión y caos digital.
\item Describir los detalles del algoritmo criptográfico con base a los requerimientos básicos para sistemas criptográficos basados en caos.   
\item Simular en MatLab el algoritmo de cifrado propuesto para la protección de imagen a color RGB, mostrar su flexibilidad para el cifrado de distintas imágenes y realizar su respectivo análisis de seguridad. 
\item Implementar el algoritmo criptográfico en un sistema embebido, con base a un microcontrolador de 32 bits para el cifrado de texto alfanumérico y para el cifrado de datos biométricos (plantilla dactilar).
\item Evaluar la seguridad del cifrado caótico a nivel software (lógico) en cada implementación embebida, con análisis estadísticos, diferenciales y de desempeño, para determinar su potencial uso en aplicaciones embebidas.
\end{enumerate}

El alcance de la presente tesis doctoral aplica de forma teórica y práctica. Se diseña un algoritmo criptográfico caótico con altas características de seguridad y eficiencia, lo cual, se verifica con un amplio análisis de seguridad; por otra parte, el algoritmo se implementa en tecnología digital de microcontrolador para aplicaciones embebidas. El algoritmo criptográfico propuesto es utilizable para la transmisión y almacenamiento de información de forma segura.  \\

Los datos cifrados son recolectados del microcontrolador mediante memoria USB para realizar los distintos análisis de seguridad, con base a simulaciones en MatLab. \\

Las contribuciones de este trabajo de investigación con relación a los objetivos mencionados, se pueden consultar en \textbf{Productos derivados de este trabajo doctoral} (Sec.~10.3).

\section{Organización del manuscrito}
Este trabajo de tesis doctoral esta compuesto por diez capítulos. En los siguientes párrafos se describe de manera breve el contenido en cada uno de ellos.
\begin{itemize}
\item \textbf{Capítulo 1:} se presenta la introducción de este trabajo de investigación, la motivación y los objetivos a alcanzar.  
\item \textbf{Capítulo 2:} en este capítulo, se describe al caos desde una perspectiva matemática. También, se presentan las propiedades de un sistema caótico. En particular, se describen dos sistemas caóticos: sistema de Lorenz y mapa logístico. En cada caso, se verifica el comportamiento caótico determinando el máximo exponente de Lyapunov, con base a simulaciones en MatLab. 
\item \textbf{Capítulo 3:} se presenta breve historia e introducción a la criptografía. Se describen los componentes de un sistema criptográfico y muestra su clasificación. Los datos cifrados deben cumplir con cierto grado de aleatoriedad de acuerdo con el estándar FIPS 140-2, los cuales, se menciona en este capítulo. Se muestran los dos tipos de cifrado caótico propuestos en la literatura: analógico y digital. Además, se describen los requerimientos y reglas básicas que un sistema criptográfico basado en caos digital debe cumplir.
\item \textbf{Capítulo 4:} se proporciona una introducción a los sistemas embebidos, sus componentes y sus aplicaciones. Se presenta el microcontrolador utilizado en este trabajo de investigación. Los problemas de seguridad en sistemas embebidos complejos se presentan para motivar el uso de criptografía caótica. También, se presenta una revisión bibliográfica sobre la implementación de sistemas caóticos y criptográficos en microcontroladores.
\item \textbf{Capítulo 5:} se reporta la implementación en microcontrolador del sistema de Lorenz y mapa logístico con programación en lenguaje C. Además, la existencia de caos se verifica con un estudio de exponentes de Lyapunov. También, se presenta la eficiencia y los recursos de implementación en cada caso.
\item \textbf{Capítulo 6:} se presenta el algoritmo de cifrado propuesto y las consideraciones matemáticas para generar un cifrado con excelentes características de aleatoriedad para que pueda resistir los distintos ataques criptoanalíticos y para que sea eficiente.
\item \textbf{Capítulo 7:} en este capítulo, se muestra la implementación en MatLab del algoritmo de cifrado propuesto, para \textit{imágenes a color RGB}. Un análisis de seguridad mediante estadísticas valida la calidad del cifrado y una comparación con otro algoritmo lo ratifica.
\item \textbf{Capítulo 8:} se presenta la implementación en microcontrolador del cifrado propuesto, para \textit{texto alfanumérico}. La calidad del cifrado se valida con una serie de análisis de seguridad  a nivel lógico (estadístico). Además, se presenta su robustez ante ataques poderosos reportados en la literatura, el desempeño y los recursos de implementación para validar su aplicación en sistemas embebidos.
\item \textbf{Capítulo 9:} se presenta la implementación en microcontrolador del cifrado propuesto, para \textit{plantilla de huella dactilar} con aplicación a sistemas embebidos de control de accesos físico o lógico. Un análisis completo de seguridad validan el cifrado caótico propuesto y en este caso, se muestra la calidad criptográfica con un análisis de aleatoriedad. Finalmente, la implementación digital se valida con un análisis de recursos físicos y eficiencia de cifrado.
\item \textbf{Capítulo 10:} se mencionan las conclusiones de este trabajo doctoral de manera general, las publicaciones que generó este trabajo de investigación doctoral, las contribuciones más sobresalientes y algunas vertientes para trabajos futuros. 
\end{itemize}
 


